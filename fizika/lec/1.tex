\section*{Лекция 1}
Магнитное поле - это особый вид материи, который создается движущимися зарядами и действут на движ. в этом поле электрические заряды.
\[[B] = \text{Тл}\]
Сила Лоренца $$ \vec{F} = q\vec{E} + q[\vec{v} \vec{B}] $$
\section*{Магнитное поле в веществе}
Процесс изменения состояния магнетика во внешнем магнитном поле называют \par\textbf{намогничиванием}.
$$ \mu = \dfrac{\vec{B}}{\vec{B_0}} \text{ - магнитная проницаемость вещества. }$$
\textbf{Магнетики: }
\renewcommand{\labelenumii}{\arabic{enumi}.\arabic{enumii}.}
\begin{enumerate}
    \item Диамагнетики ($\mu < 1$)
    \item Парамагнетики ($\mu > 1$)
    \item Фарромагнетики ($\mu >> 1$)
\end{enumerate}
Доменная структура - это остаточная намогниченность.\par
\section*{Электромагнитная индукция}
\textbf{Если поток вектора магнитной индукции, пронизывающий замкнутый, проводящий контур меняется, то в контуре возникает электрический ток (индукционный ток).}

\vspace{1em}

Потоком вектора магнитной индукции (магнитным потоком) через малую поверхность площадью $dS$ называется скалярная физическая величина, равная

\[
d\Phi = \vec{B} \cdot d\vec{S}
\]

\noindent где
\[
d\vec{S} = dS \cdot \vec{n}
\]
\[
\vec{n} \text{ --- единичный вектор нормали к площади.}
\]

\vspace{1em}

\noindent Учитывая угол $\alpha$ между $\vec{B}$ и $\vec{n}$:
\[
d\Phi = B \, dS \cos\alpha
\]
\par
-- \textbf{Увеличение потока} $\dfrac{d\Phi}{dt} > 0$ вызывает $E < 0$, т.е. индукционное поле $B_i$ направлено навстречу внешнему полю, поток которого $\Phi_B$. \par
-- \textbf{Уменьшение потока} $\dfrac{d\Phi}{dt} < 0$ вызывает $E > 0$, т.е. совпадает с направлением внешнего поля, поток которого $\Phi_B$.
\pagebreak
\section*{Важное замечание}

\textbf{ВАЖНО: Закон Фарадея универсален, так как не зависит от способа изменения магнитного поля.}

\vspace{1em}

\[
\mathcal{E} = -\frac{d\Phi_B}{dt} = -\frac{d(B S)}{dt} = -\frac{d(B S \cos \alpha)}{dt}
\]

\vspace{1em}

\textbf{Поток магнитной индукции можно менять следующими способами:}

\begin{enumerate}
    \item Изменять площадь рамки.
    \item Вращать рамку.
    \item Изменять внешнее магнитное поле.
\end{enumerate}

