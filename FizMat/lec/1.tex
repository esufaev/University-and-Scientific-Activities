\section*{Лекция 1. Классификация и приведение к каноническому виду диференциальных уравнений 2 порядка.}

\begin{definition}
    Уравнением в частных производных 2 порядка называют
    $$ \sum_{i, j = 1}^{n} a_{ij} \dfrac{\delta ^ 2 u}{\delta x_i \delta x_j} + \sum_{i = 1}^{n} b_i \dfrac{\delta u}{\delta x_i} + cu + f(x) = 0 $$
    $$ u = u(x_1, \cdots, x_n), c = c(x), a_{ij} = a_{ij} (x), b_i = b_i(x) $$
\end{definition}
Если $ u, c, a_{ij}, b_{i} = \text{const} \quad \forall i, j $, то
уравнение имеет пстоянные коэфициенты. \par
Рассмотрим случай $n = 2$:
$$ a_{11} (x, y) \dfrac{\delta ^ 2 u}{\delta x^2} + 2 a_{12}(x, y) \dfrac{\delta ^ 2 u}{\delta x \delta y} + a_{22} (x, y) \dfrac{\delta ^ 2 u}{\delta y^2} + F(f(x, y), \dfrac{\delta u}{\delta x}, \dfrac{\delta u}{\delta y}, u) = 0 $$
\begin{remark}
    Принято писать $\dfrac{\delta u}{\delta x} = u_x, \dfrac{\delta ^ 2 u}{\delta x \delta y} = u_{xy}, \dfrac{\delta ^ 2 u}{\delta x ^ 2} = u_{xx}, \cdots. $
\end{remark}

Сделаем замену переменных: $ \xi = \phi (x, y), \quad \eta = \psi (x, y) $. И так как они линейно независимы:
$$ \mathbb{J} = \begin{vmatrix}
        \dfrac{\delta \phi}{\delta x} & \dfrac{\delta \phi}{\delta y} \\
        \dfrac{\delta \psi}{\delta x} & \dfrac{\delta \psi}{\delta y}
    \end{vmatrix} \neq 0 $$
$$ u_x = \xi_x u_\xi + \eta_x u_\eta $$
$$ u_{xx} = (\xi_x u_\xi + \eta_x u_\eta)_x = u_{\xi \xi} \xi^2_x + 2u_{\xi \eta} \xi_x \eta_x + u_{\eta \eta} \eta^2_x + u_\xi \xi_{xx} + u_\eta \eta_{xx} $$
$$ u_{yy} = (\xi_y u_\xi + \eta_y u_\eta)_y = u_{\xi \xi} \xi^2_y + 2u_{\xi \eta} \xi_y \eta_y + u_{\eta \eta} \eta^2_y + u_\xi \xi_{y} + u_\eta \eta_{yy} $$
$$ u_{xy} = (u_\xi \xi_x + u_\eta \eta_x)_y = \cdots. $$
Подставляя это в исходное уравнение, мы получим:
$$ u_{\xi \xi}' \overbrace{[a_{11}(\xi_x)^2 + a_{12} \xi_x \xi_y + a_{22} (\xi_y)^2]}^{\tilde{a}_{11}} + u_{\eta \eta} \overbrace{[a_{11} (\eta_x)^2 + 2 a_{12} \eta_x \eta_y + a_{22} (\eta_y)^2]}^{\tilde{a}_{22}} + $$
$$ + u_{\eta \xi} \overbrace{[2 a_{11} \xi_x \eta_x + 2 a_{22} \xi_y \eta_y + 2a_{12} ( \xi_x \eta_y + \xi_y \eta_x )]}^{\tilde{a}_{12}} + \tilde{F} (\xi, \eta, u, u_\xi, u_\eta) = 0 $$
Итого, получаем:
$$ a_{11} (\xi_x)^2 + 2 a_{12} \xi_x \xi_y + a_{22} (\xi_y)^2 = 0 $$
$$ a_{11} \left(\dfrac{\xi_x}{\xi_y}\right)^2 + 2 a_{12} \left(\dfrac{\xi_x}{\xi_y}\right) + a_{22} = 0 $$
Пусть $ \xi(x, y) = \text{const} $, выразим тогда $ y = y(x) $. Отсюда $ d\xi = \xi_x dx + \xi_y dy \Rightarrow \dfrac{dy}{dx} = -\dfrac{\xi_x}{\xi_y} $
Значит имеем:
$$ a_{11} \left(\dfrac{dy}{dx}\right)^2 + 2a_{12} \left(\dfrac{dy}{dx}\right) + a_{22} = 0 $$
\begin{definition}
    $a_{11} (dy)^2 - 2 a_{12} dy dx + a_{22} (dx)^2 = 0$ - характеристическое уравнение исходного уравнения. 
\end{definition}
$$ \dfrac{dy}{dx} = \dfrac{a_{12} \pm \overbrace{\sqrt{(a_{12})^2 - a_{11} a_{22}}}^{D}}{a_{11}} $$
\begin{enumerate}
    \item D > 0 - гиперболический тип;
    \item D = 0 - параболический тип;
    \item D < 0 - эллиптический тип.
\end{enumerate}
В 1. делаем замену $ \xi = \phi (x, y) = c, \quad \eta = \psi (x, y) = c $ \par
В 2. $\xi = \phi (x, y) = c \quad \quad \eta = \psi (x, y) = c$ - выбираем сами. Она должна быть лин. независимо от $\phi (x, y)$ \par
В 3. имеем $ \overbrace{\phi (x, y)}^{\xi} \pm i \overbrace{\psi (x, y)}^{\eta} = c $