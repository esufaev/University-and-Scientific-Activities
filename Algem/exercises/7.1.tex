\section*{Параграф 7.1}
\large{\bf{Упражнение 1.}} Пространство $\mathbb{R}^n$ превращается в вещественное евкли­дово пространство, 
если определить скалярное произведение элементов x и y
пространства $\mathbb{R}^n$ по формуле $$ (x, y) = \sum_{k = 1}^{n} x_k y_k, $$ 
Это, так называемое, стандартное скалярное произведение. Проверить, что ак­сиомы скалярного произведения выполнены.
\begin{solution}
    Первая аксиома: $$ (x, x) = \sum_{k = 1}^{n} x_k x_k = \sum_{k = 1}^{n} x_k^2 \geq 0 $$
    причем отсюда следует, что $ ((x, x) = 0) \equiv (x = 0) $ \par
    Вторая аксиома: $$ (x, y) = \sum_{k = 1}^{n} x_k y_k = \sum_{k = 0}^{n} y_k x_k = (y, x) $$ \par
    Третья аксиома: $$ (\alpha x + \beta y, z) = \sum_{k = 1}^{n} (\alpha x_k + \beta y_k) z_k = \alpha \sum_{k = 0}^{n} x_k z_k + \beta \sum_{k = 1}^{n} y_k z_k = \alpha (x, z) + \beta (y, z) $$ \par
\end{solution}

\large{\bf{Упражнение 2.}} Пространство $ \mathbb{R}^n $ также можно 
превратить в веще­ственное евклидово пространство, 
если определить скалярное произведение элементов x и y 
пространства $ \mathbb{R}^n $ по формуле $$ (x, y) = \sum_{k = 1}^{n} \rho_k x_k y_k, $$ 
где $\rho_1$, $\rho_2$, $\cdots$, $\rho_n$ — заданные положительные числа. Это, так называемое
скалярное произведение с весами. Проверить, что аксиомы скалярного произ­
ведения выполнены.
\begin{solution}
    Первая аксиома: $$ (x, x) = \sum_{k = 1}^{n} \rho_k x_k x_k = \sum_{k = 1}^{n} \rho_k x_k^2 \geq 0 $$
    причем отсюда следует, что $ ((x, x) = 0) \equiv (x = 0) $ \par
    Вторая аксиома: $$ (x, y) = \sum_{k = 1}^{n} \rho_k x_k y_k = \sum_{k = 0}^{n} \rho_k y_k x_k = (y, x) $$ \par
    Третья аксиома: $$ (\alpha x + \beta y, z) = \sum_{k = 1}^{n} \rho_k (\alpha x_k + \beta y_k) z_k = \alpha \sum_{k = 0}^{n} \rho_k x_k z_k + \beta \sum_{k = 1}^{n} \rho_k y_k z_k = \alpha (x, z) + \beta (y, z) $$ \par
\end{solution} 

\large{\bf{Упражнение 3.}} Пространство $ \mathbb{C}^n $ превращается в комплексное 
евкли­дово пространство, если определить скалярное произведение элементов x и y
пространства $ \mathbb{C}^n $, например, по формуле $$ (x, y) = \sum_{k = 1}^{n} x_k \overline{y_k}, $$
Это стандартное скалярное произведение в $ \mathbb{C}^n $. Проверить, что аксиомы ска­
лярного произведения и в этом случае выполнены.
\begin{solution}
    Первая аксиома: $$ (x, x) = \sum_{k = 1}^{n} x_k \overline{x_k} = \sum_{k = 1}^{n} |x_k^2| \geq 0 $$
    причем отсюда следует, что $ ((x, x) = 0) \equiv (x = 0) $ \par
    Вторая аксиома: $$ \overline{(y, x)} = \overline{\sum_{k = 1}^{n} y_k \overline{x_k}} = \sum_{k  1}^{n} \overline{y_k \overline{x_k}} = \sum_{k = 1}^{n} x_k \overline{y_k} = (x, y) $$ \par
    Третья аксиома: $$ (\alpha x + \beta y, z) = \sum_{k = 1}^{n} (\alpha x_k + \beta y_k) \overline{z_k} = \alpha \sum_{k = 0}^{n} x_k \overline{z_k} + \beta \sum_{k = 1}^{n} y_k \overline{z_k} = \alpha (x, z) + \beta (y, z) $$ \par 
\end{solution}\pagebreak
