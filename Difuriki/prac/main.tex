\documentclass[a4paper, 12pt]{article}

% Подключение русского языка.
\usepackage[russian]{babel}
\usepackage[utf8]{inputenc}
\usepackage[T2A]{fontenc}

% Настройка внешнего вида заголовков.
\usepackage{titlesec}
\newcommand{\chapterFont}{\fontsize{16}{15} \selectfont}
\newcommand{\chapterNumberFont}{\fontsize{14}{15} \selectfont}
\newcommand{\sectionFont}{\fontsize{14}{15} \selectfont}
\renewcommand\thesection{\thechapter.\arabic{section}.}
\titleformat{\chapter}[display]
{\normalfont \chapterFont \bfseries}{\centering \chapterNumberFont
	\chaptertitlename \ \thechapter}{10pt}{\centering \chapterFont}
\titlespacing{\chapter}{0pt}{8pt}{25pt}
\titleformat{\section}
{\normalfont \sectionFont \bfseries}{\centering \thesection}{6pt}
{\centering \sectionFont}
\titlespacing{\section}{0pt}{25pt}{5pt}
% Настройка размеров страниц.
\setlength{\parindent}{0pt} \renewcommand{\baselinestretch}{1.2} \topmargin = 0mm \textwidth =
175mm \textheight = 260mm \hoffset = -17.5mm \voffset = -25.5mm
% Подключение библиотек для изображений.
\usepackage{tcolorbox}
\usepackage{graphicx}
\usepackage{float}
% Подключение библиотек для математических символов.
\usepackage{amsthm}
\usepackage{parcolumns}
\usepackage{amsmath}
\usepackage{amsfonts}
% Настройка внешнего вида для теорем, утверждений и т.д.
\theoremstyle{definition}
\newtheorem*{theorem}{Теорема}
\newtheorem*{definition}{Определение}
\newtheorem*{example}{Пример}
\newtheorem*{remark}{Замечание}
\newtheorem*{solution}{Решение}

\begin{document}

\section*{Задача 1104}
Найти общее решение дифференциального уравнения:
$$(1 - x)y'' - 2y' + y = 0$$

\begin{solution}
   Решим задачу методом рядов.

   1) Предположим, что решение имеет вид:
   $$y = \sum_{k = 0}^{\infty} y_k x^k$$

   2) Тогда:
   $$y' = \sum_{k = 0}^{\infty} y_k k x^{k-1}$$
   $$y'' = \sum_{k = 0}^{\infty} y_k k(k-1) x^{k-2}$$

   3) Подставляем в исходное уравнение:
   $$\sum_{k = 0}^{\infty} y_k k(k-1)x^{k-2} - y_k k(k-1)x^{k-1} - 2y_k kx^{k-1} + y_k x^k = 0$$

   4) Приравнивая коэффициенты при $x^k$:
   $$y_k - k(k+1)y_{k+1} - 2(k+1)y_{k+1} + y_{k+2}(k+1)(k+2) = 0$$

   5) Рекуррентное соотношение:
   $$y_{k+2} = \frac{y_{k+1}(k+1)(k+2) - y_k}{(k+1)(k+2)}$$

   6) Два линейно независимых решения:

   Первое решение ($y_0 = 1, y_1 = 0$):
   $$y_1 = 1 - \frac{1}{2}x - \frac{1}{2}x^2 - \frac{11}{24}x^3 + \cdots$$

   Второе решение ($y_0 = 0, y_1 = 1$):
   $$y_2 = x + \frac{5}{6}x^2 + \frac{3}{4}x^3 + \cdots$$

   Таким образом, общее решение:
   $$y = c_1y_1 + c_2y_2$$
   где $c_1$ и $c_2$ — произвольные константы.
\end{solution}
\section*{Задача 1106}
Найти общее решение дифференциального уравнения:
$$y'' - xy' + xy = 0$$

\begin{solution}
1) Предположим решение в виде степенного ряда:
   $$y = \sum_{k = 0}^{\infty} y_k x^k$$

2) Производные:
   $$y' = \sum_{k = 0}^{\infty} y_k k x^{k-1}$$
   $$y'' = \sum_{k = 0}^{\infty} y_k k(k-1) x^{k-2}$$

3) Подставляем и приравниваем коэффициенты:
   $$y_{k+3} = \frac{y_{k+1}(k+1) - y_k}{(k+3)(k+2)}$$

4) Три линейно независимых решения:

   Первое решение ($y_0 = 1, y_1 = 0, y_2 = 0$):
   $$y_1 = 1 - \frac{1}{6}x^3 - \frac{1}{40}x^5 + \cdots$$

   Второе решение ($y_0 = 0, y_1 = 1, y_2 = 0$):
   $$y_2 = x + \frac{1}{6}x^3 - \frac{1}{12}x^4 + \frac{1}{40}x^5 + \cdots$$

   Третье решение ($y_0 = 0, y_1 = 0, y_2 = 1$):
   $$y_3 = x^2 + \frac{1}{6}x^4 - \frac{1}{40}x^5 + \cdots$$

Таким образом, общее решение:
$$y = c_1y_1 + c_2y_2 + c_3y_3$$
где $c_1$, $c_2$ и $c_3$ — произвольные константы.
\end{solution}

\pagebreak\section*{Задача 1109}
Найти общее решение дифференциального уравнения: $$ y''' - xy'' + (x - 2)y' + y = 0 $$
\begin{solution}
    1) При $x_0 = 0 $, $ p_0(0) \neq 0 $ => $ y = \sum_{k = 0}^{\inf} y_k x^k $. \par
    2) Преобразуем: $$ \sum_{k = 0}^{\inf} \underset{k = p + 3}{k (k - 1) (k - 2) y_k x^{k - 3}} - \sum_{k = 0}^{\inf} \underset{k = p + 1}{k (k - 1) y_k x^{k -2 + 1}} + \sum_{k = 0}^{\inf} \underset{k = p}{k y_k x^{k - 1 + 1}} - 2 \sum_{k = 0}^{\inf} \underset{k = p + 1}{k y_k x^{k - 1}} + \sum_{k = 0}^{\inf} \underset{k = p}{y_k x^k} = 0 $$
    При $ x^p $ : $$ (p + 3) (p + 2) (p + 1) y_{p + 3} - (p + 1) p y_{p + 1} + p y_p - 2 (p + 1) + y_p = 0 $$
    Сократим $ (p + 1) $ : $$ (p + 3) (p + 2) y_{p + 3} - (p + 2) y_{p + 1} + y_p = 0 $$
    3) Выразим $ y $ : $$ y_{p + 3} = \frac{y_{p + 1}}{p + 3} - \frac{y_p}{(p + 3)(p + 2)} $$
    4) Построим единичную матрицу: \par
    $$\begin{array}{c ccc}
                      & 1) & 2) & 3) \\
                y_0 = & 1  & 0  & 0  \\
                y_1 = & 0  & 1  & 0  \\
                y_2 = & 0  & 0  & 1  \\
            \end{array}$$\par
    \quad1) $ y_3 = 0 - \frac{1}{6} = -\frac{1}{6} $, $ y_4 = 0 $, $ y_1 = 1 - \frac{x^3}{6} + 0 + \cdots $\par
    \quad2) $ y_3 = \frac{1}{3} - 0 = \frac{1}{3} $, $ y_4 = -\frac{1}{12} $, $ y_2 = x + \frac{x^3}{3} - \frac{x^4}{12} + \cdots $\par
    \quad3) $ y_3 = 0 $, $ y_4 = \frac{1}{4} $, $ y_3 = x^2 + \frac{x^4}{4} + \cdots $\par
    5) Таким образом общее решение: $$ y = c_1y_1 + c_2y_2 + c_3y_3 $$ где $c_1$, $c_2$ и $c_3$ - произвольные константы.

\end{solution}\pagebreak

\end{document}
